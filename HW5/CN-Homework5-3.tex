% ----------------------------------------------------------------
% AMS-LaTeX Paper ************************************************
% **** -----------------------------------------------------------
\documentclass{amsart}
\usepackage{graphicx}
% ----------------------------------------------------------------
\vfuzz2pt % Don't report over-full v-boxes if over-edge is small
\hfuzz2pt % Don't report over-full h-boxes if over-edge is small
% THEOREMS -------------------------------------------------------
\newtheorem{thm}{Theorem}[section]
\newtheorem{cor}[thm]{Corollary}
\newtheorem{lem}[thm]{Lemma}
\newtheorem{prop}[thm]{Proposition}
\theoremstyle{definition}
\newtheorem{defn}[thm]{Definition}
\theoremstyle{remark}
\newtheorem{rem}[thm]{Remark}
\numberwithin{equation}{section}
% MATH -----------------------------------------------------------
\newcommand{\norm}[1]{\left\Vert#1\right\Vert}
\newcommand{\abs}[1]{\left\vert#1\right\vert}
\newcommand{\set}[1]{\left\{#1\right\}}
\newcommand{\Real}{\mathbb R}
\newcommand{\eps}{\varepsilon}
\newcommand{\To}{\longrightarrow}
\newcommand{\BX}{\mathbf{B}(X)}
\newcommand{\A}{\mathcal{A}}
% ----------------------------------------------------------------
\begin{document}

\title{Complex Networks  - Spring 2025\\{\bf Homework 5}}%
\author{Instructor: Jia Liu}%
\date{}

%\dedicatory{}%
%\commby{}%
% ----------------------------------------------------------------

\maketitle
% ----------------------------------------------------------------
\begin{itemize}
\item DUE on  04/30/2025 11:59pm C.T.
\item You must finish the homework independently. You may discuss them with your team members and me. 
\item Please name your file as follows: $LastnameInitials-MAP5990quiz1.pdf$.If your name is Alan David Roberts, file name is $RobertsAD-MAP5990quiz1.pdf$.
\item Try to keep the file size less than 4MB.
\item You can resubmit the quiz if you want. Please specify which one is the one to be graded. Otherwise I will grade the most recent version.
\item DO NOT EMAIL me the quiz. All quizzes are submitted via Canvas.
\end{itemize}


\clearpage
\begin{enumerate}

%---------------------------------------------------------------------------------

\item {\bf These questions will be great preparations for your final project report. You can discuss with your team members but each of you must submit the solutions by yourself.}
\begin{enumerate}
\item Introduce the complex network you will report in your final project:
\begin{enumerate}
\item What is the name of your network?
\item What is the type of the network? Social, biology, information, www, etc?
\item Explain why this network is important.
\item What is the network structure of your network such as node numbers, edge numbers. 
\end{enumerate}

\vspace{1cm}

\item 
For your references, please include a brief summary of the five references you have cited in the midterm report. Why you want to include these references? What is each reference about? Any past research contribution or related work?
\vspace{1cm}

\item For your final project, what is the methodology you want to use for this project? Explain why you choose this methodology. Keep in mind, here is about the methodology (such as centrality, community detection method, not the experimental steps or network introduction). 
\vspace{1cm}
\item Summary what you have learned from this project. Any future work? 
\vspace{1cm}
\end{enumerate}

\clearpage

\item Download the power grid network from Neuman's website or Gephi wiki. You may download it from the assignment page. 
\begin{enumerate}
\item Use Gephi to plot the network. Make sure to use centrality and communities so that you can show the properties of the network.
\item choose two different layouts in Gephi to plot the network. 
\item Export the plots and submit the two plots with different layouts. Make sure to use the centrality and communities to show the properties of the network in each plot.
\item Use the resolution 1.0 for the community detection. How many communities you have? 
\item Change the resolution to 0.5 and 5.0, how many communities do you  have for each case?
\end{enumerate}
\vspace{1cm}

\item  Repeat the same steps in the previous problem on the network from your final project. Make sure to use centrality and communities so that you can show the properties of the network.
% ----------------------------------------------------------------

\end{enumerate}

\end{document}
% ----------------------------------------------------------------
