% CVPR 2023 Paper Template
% based on the CVPR template provided by Ming-Ming Cheng (https://github.com/MCG-NKU/CVPR_Template)
% modified and extended by Stefan Roth (stefan.roth@NOSPAMtu-darmstadt.de)

\documentclass[10pt,twocolumn,letterpaper]{article}

%%%%%%%%% PAPER TYPE  - PLEASE UPDATE FOR FINAL VERSION
% \usepackage[review]{cvpr}      % To produce the REVIEW version
\usepackage{cvpr}              % To produce the CAMERA-READY version
%\usepackage[pagenumbers]{cvpr} % To force page numbers, e.g. for an arXiv version

% Include other packages here, before hyperref.
\usepackage{graphicx}
\usepackage{amsmath}
\usepackage{amssymb}
\usepackage{booktabs}


% It is strongly recommended to use hyperref, especially for the review version.
% hyperref with option pagebackref eases the reviewers' job.
% Please disable hyperref *only* if you encounter grave issues, e.g. with the
% file validation for the camera-ready version.
%
% If you comment hyperref and then uncomment it, you should delete
% ReviewTempalte.aux before re-running LaTeX.
% (Or just hit 'q' on the first LaTeX run, let it finish, and you
%  should be clear).
\usepackage[pagebackref,breaklinks,colorlinks]{hyperref}


% Support for easy cross-referencing
\usepackage[capitalize]{cleveref}
\crefname{section}{Sec.}{Secs.}
\Crefname{section}{Section}{Sections}
\Crefname{table}{Table}{Tables}
\crefname{table}{Tab.}{Tabs.}


%%%%%%%%% PAPER ID  - PLEASE UPDATE
\def\cvprPaperID{*****} % *** Enter the CVPR Paper ID here
\def\confName{CVPR}
\def\confYear{2023}


\begin{document}

%%%%%%%%% TITLE - PLEASE UPDATE
\title{\LaTeX\ Author Guidelines for \confName~Proceedings}

\author{Renan Monteiro Barbosa\\
University of West Florida\\
11000 University Parkway, Pensacola, FL 32514, United States of America \\
{\tt\small rmb54@students.uwf.edu}
% For a paper whose authors are all at the same institution,
% omit the following lines up until the closing ``}''.
% Additional authors and addresses can be added with ``\and'',
% just like the second author.
% To save space, use either the email address or home page, not both
\and
Bala Raju Pidatala\\
University of West Florida\\
11000 University Parkway\\
{\tt\small bp94@students.uwf.edu}
\and
Chantal Ojurongbe\\
University of West Florida\\
11000 University Parkway\\
{\tt\small czo1@students.uwf.edu}
}
\maketitle

%%%%%%%%% ABSTRACT
\begin{abstract}
   The ABSTRACT is to be in fully justified italicized text, at the top of the left-hand column, below the author and affiliation information.
   Use the word ``Abstract'' as the title, in 12-point Times, boldface type, centered relative to the column, initially capitalized.
   The abstract is to be in 10-point, single-spaced type.
   Leave two blank lines after the Abstract, then begin the main text.
   Look at previous CVPR abstracts to get a feel for style and length.
\end{abstract}

%%%%%%%%% BODY TEXT

% ##############################################################################
% Introduction
% ##############################################################################

\section{Introduction}
\label{sec:intro}

\textbf{The Criticality and Fragility of Power Grids}

Modern civilization is intrinsically linked to and deeply reliant upon the continuous and reliable supply of electrical power. Power grids function as the operational backbone for nearly all facets of contemporary life, underpinning economic stability, ensuring national security, supporting public health systems, and enabling essential daily activities \cite{number1}(Add a citation number 1). These vast networks are not merely independent entities; they are foundational critical infrastructures upon which other vital systems, such as telecommunications, transportation, water supply, and financial services, depend.(add a citation number 5) The interconnected nature of modern infrastructure means that disruptions within the power grid can cascade, causing widespread societal and economic paralysis.

Despite their criticality, power grids exhibit inherent fragility. Historical events serve as stark reminders of the potential consequences of grid failures. Large-scale blackouts, such as the 2021 Texas winter storm event (add a citation number 8) and the 2003 Northeast blackout (add a citation number 10), have left millions without power, often under hazardous conditions. The economic repercussions are staggering, with individual events potentially costing billions of dollars in direct damages and lost productivity.(Add a citation number 1) Beyond economic costs, the human toll is significant. Power outages compromise public health by disabling heating and cooling systems during extreme temperatures, disrupting access to essential medical services, and causing spoilage of food and medication reliant on refrigeration.(Add a citation number 1) The 2021 Texas failure, for instance, not only caused widespread hardship but also brought the entire state grid perilously close—within minutes—to a complete collapse that could have taken weeks or months to restore, highlighting the potential for catastrophic, long-term disruptions.(Add a citation number 8) Similarly, the devastation following Hurricane Maria in Puerto Rico underscored the prolonged societal impact of grid failure, with extended outages contributing to significant excess mortality and immense economic loss.(Add a citation number 10)

\textbf{Escalating Threats and Increasing Complexity}

The challenges facing power grid reliability are intensifying due to a confluence of escalating threats and growing system complexity. Climate change is driving an increase in the frequency and intensity of extreme weather events, which are a primary cause of power outages.(Add a citation number 9) Hurricanes, severe snow and ice storms, heatwaves, wildfires, and heavy rainfall events place immense physical stress on grid infrastructure, often exceeding the design limits of aging components.(Add a citation number 1) The 2021 Texas freeze (Add a citation number 8) and widespread summer heatwaves straining grids across the US (Add a citation number 10) exemplify this vulnerability. Aging transmission lines, transformers, and substations, many operating beyond their intended lifespans, are less resilient to these weather-related stresses.(Add a citation number 1)

Simultaneously, the cyber threat landscape is evolving rapidly. Power grids are increasingly reliant on digital control systems (e.g., SCADA) and interconnected technologies, creating new avenues for malicious actors.(Add a citation number 2) Cyber-attacks targeting grid operations are growing in sophistication and frequency, with state-sponsored actors and criminal groups demonstrating the capability to disrupt or damage grid components.(Add a citation number 9) The integration of renewable energy sources, while beneficial for sustainability, introduces new potential vulnerabilities through distributed energy resources (DERs) like solar inverters and their associated communication networks, which may lack the robust security protocols of traditional, centralized power plants.(Add a citation number 14)

Physical threats also pose a significant and growing risk. Intentional attacks on substations and transmission infrastructure, including acts of sabotage, vandalism, and gunfire, have increased markedly in recent years.(Add a citation number 14) Copper theft from substations remains a persistent problem, causing damage and outages.(Add a citation number 15) Inadequate physical security measures and the vast, often remote, expanse of grid infrastructure make physical protection challenging.(Add a citation number 15)

These threats do not exist in isolation; they can interact and compound, creating scenarios far more damaging than standalone events. Extreme weather, for example, can physically stress the grid while simultaneously creating chaotic conditions that malicious actors might exploit for cyber-attacks, knowing the system is already vulnerable and response capabilities are strained.(Add a citation number 20) Studies simulating such compound cyber-physical threats have shown significantly amplified impacts, with unmet electricity demand potentially tripling compared to a cyber-attack alone.(Add a citation number 20) Furthermore, physical vulnerabilities, such as poorly maintained infrastructure or inadequate site security, can directly enable cyber intrusions by providing physical access to supposedly secure network components.(Add a citation number 15) This interplay necessitates a vulnerability assessment approach that considers the potential synergy between different threat vectors.

The inherent complexity of the grid itself magnifies these vulnerabilities. Power systems are vast, interconnected networks spanning large geographical areas, often operated by multiple entities.(Add a citation number 21) This interconnectedness, while providing operational flexibility, also creates pathways for failures to cascade rapidly across the system.(Add a citation number 6) The ongoing energy transition, involving the integration of diverse and often intermittent renewable energy sources and the deployment of smart grid technologies, further increases operational complexity.(Add a citation number 14) While modernization aims to enhance efficiency and control, it paradoxically introduces new types of interdependencies and potential failure points, particularly in the cyber domain.(Add a citation number 2) This suggests a critical trade-off where technological advancements, if not implemented with integrated security and resilience considerations, can inadvertently expand the system's attack surface.

%------------------------------------------------------------------------
% ##############################################################################
% Methods
% ##############################################################################

\section{Methods}
\label{sec:methods}

All text must be in a two-column format.
The total allowable size of the text area is $6\frac78$ inches (17.46 cm) wide by $8\frac78$ inches (22.54 cm) high.
Columns are to be $3\frac14$ inches (8.25 cm) wide, with a $\frac{5}{16}$ inch (0.8 cm) space between them.


%------------------------------------------------------------------------
% ##############################################################################
% Results
% ##############################################################################

\section{Results}
\label{sec:results}

All text must be in a two-column format.
The total allowable size of the text area is $6\frac78$ inches (17.46 cm) wide by $8\frac78$ inches (22.54 cm) high.
Columns are to be $3\frac14$ inches (8.25 cm) wide, with a $\frac{5}{16}$ inch (0.8 cm) space between them.
The main title (on the first page) should begin 1 inch (2.54 cm) from the top edge of the page.
The second and following pages should begin 1 inch (2.54 cm) from the top edge.
On all pages, the bottom margin should be $1\frac{1}{8}$ inches (2.86 cm) from the bottom edge of the page for $8.5 \times 11$-inch paper;
for A4 paper, approximately $1\frac{5}{8}$ inches (4.13 cm) from the bottom edge of the
page.

%------------------------------------------------------------------------
% ##############################################################################
% Conclusion
% ##############################################################################
\section{Conclusion}

You must include your signed IEEE copyright release form when you submit your finished paper.
We MUST have this form before your paper can be published in the proceedings.

Please direct any questions to the production editor in charge of these proceedings at the IEEE Computer Society Press:
\url{https://www.computer.org/about/contact}.

\cite{Authors14}

\cite{Alpher02,Alpher03,Alpher05,Authors14b,Authors14}

%%%%%%%%% REFERENCES
{\small
\bibliographystyle{ieee_fullname}
\bibliography{egbib}
}

\end{document}
